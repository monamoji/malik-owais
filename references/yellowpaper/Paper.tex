\documentclass[9pt,oneside]{amsart}
%\usepackage{tweaklist}
\usepackage{cancel}
\usepackage{xspace}
\usepackage{graphicx}
\usepackage{multicol}
\usepackage{subfig}
\usepackage{amsmath}
\usepackage{amssymb}
\usepackage[a4paper,width=170mm,top=18mm,bottom=22mm,includeheadfoot]{geometry}
\usepackage{booktabs}
\usepackage{array}
\usepackage{verbatim}
\usepackage{caption}
\usepackage{natbib}
\usepackage{float}
\usepackage{pdflscape}
\usepackage{mathtools}
\usepackage[usenames,dvipsnames]{xcolor}
\usepackage{afterpage}
\usepackage{tikz}
\usepackage[bookmarks=true, unicode=true, pdftitle={Ethereum Yellow Paper: a formal specification of Ethereum, a programmable blockchain}, pdfauthor={Dr. Gavin Wood},pdfkeywords={Ethereum, Yellow Paper, blockchain, virtual machine, cryptography, decentralised, singleton, transaction, generalised},pdfborder={0 0 0.5 [1 3]}]{hyperref}
%,pagebackref=true

\usepackage{tabu} %requires array.

%This should be the last package before \input{Version.tex}
\PassOptionsToPackage{hyphens}{url}\usepackage{hyperref}
% "hyperref loads the url package internally. Use \PassOptionsToPackage{hyphens}{url}\usepackage{hyperref} to pass the option to the url package when it is loaded by hyperref. This avoids any package option clashes." Source: <https://tex.stackexchange.com/questions/3033/forcing-linebreaks-in-url/3034#comment44478_3034>.
% Note also this: "If the \PassOptionsToPackage{hyphens}{url} approach does not work, maybe it's "because you're trying to load the url package with a specific option, but it's being loaded by one of your packages before that with a different set of options. Try loading the url package earlier than the package that requires it. If it's loaded by the document class, try using \RequirePackage[hyphens]{url} before the document class." Source: <https://tex.stackexchange.com/questions/3033/forcing-linebreaks-in-url/3034#comment555944_3034>.
% For more information on using the hyperref package, refer to e.g. https://en.wikibooks.org/w/index.php?title=LaTeX/Hyperlinks&stable=0#Hyperlink_and_Hypertarget.

\makeatletter
 \newcommand{\linkdest}[1]{\Hy@raisedlink{\hypertarget{#1}{}}}
\makeatother
\usepackage{seqsplit}

% For formatting
%\usepackage{underscore}
%\usepackage{lipsum} % to generate filler text for testing of document rendering
\usepackage[english]{babel}
\usepackage[autostyle]{csquotes}
\MakeOuterQuote{"}

\usepackage[final]{microtype} % https://tex.stackexchange.com/questions/75140/is-it-possible-to-make-latex-mark-overfull-boxes-in-the-output#comment382776_75142

\input{Version.tex}
% Default rendering options
\definecolor{pagecolor}{rgb}{1,0.98,0.9}
\def\YellowPaperVersionNumber{unknown revision}
\IfFileExists{Options.tex}{\input{Options.tex}}

\newcommand{\hcancel}[1]{%
    \tikz[baseline=(tocancel.base)]{
        \node[inner sep=0pt,outer sep=0pt] (tocancel) {#1};
        \draw[black] (tocancel.south west) -- (tocancel.north east);
    }%
}%


\DeclarePairedDelimiter{\ceil}{\lceil}{\rceil}
\newcommand*\eg{e.g.\@\xspace}
\newcommand*\Eg{e.g.\@\xspace}
\newcommand*\ie{i.e.\@\xspace}
%\renewcommand{\itemhook}{\setlength{\topsep}{0pt}  \setlength{\itemsep}{0pt}\setlength{\leftmargin}{15pt}}

\title{Ethereum: A Secure Decentralised Generalised Transaction Ledger \\ {\smaller \textbf{Byzantium version \YellowPaperVersionNumber}}}
\author{
    Dr. Gavin Wood\\
    Founder, Ethereum \& Parity\\
    gavin@parity.io
}
\begin{document}

\pagecolor{pagecolor}

\begin{abstract}
The blockchain paradigm when coupled with cryptographically-secured transactions has demonstrated its utility through a number of projects, with Bitcoin being one of the most notable ones. Each such project can be seen as a simple application on a decentralised, but singleton, compute resource. We can call this paradigm a transactional singleton machine with shared-state.

Ethereum implements this paradigm in a generalised manner. Furthermore it provides a plurality of such resources, each with a distinct state and operating code but able to interact through a message-passing framework with others. We discuss its design, implementation issues, the opportunities it provides and the future hurdles we envisage.
\end{abstract}

\maketitle

\setlength{\columnsep}{20pt}
\begin{multicols}{2}

\section{Introduction}\label{sec:introduction}

With ubiquitous internet connections in most places of the world, global information transmission has become incredibly cheap. Technology-rooted movements like Bitcoin have demonstrated through the power of the default, consensus mechanisms, and voluntary respect of the social contract, that it is possible to use the internet to make a decentralised value-transfer system that can be shared across the world and virtually free to use. This system can be said to be a very specialised version of a cryptographically secure, transaction-based state machine. Follow-up systems such as Namecoin adapted this original ``currency application'' of the technology into other applications albeit rather simplistic ones.

Ethereum is a project which attempts to build the generalised technology; technology on which all transaction-based state machine concepts may be built. Moreover it aims to provide to the end-developer a tightly integrated end-to-end system for building software on a hitherto unexplored compute paradigm in the mainstream: a trustful object messaging compute framework.

\subsection{Driving Factors} \label{ch:driving}

There are many goals of this project; one key goal is to facilitate transactions between consenting individuals who would otherwise have no means to trust one another. This may be due to geographical separation, interfacing difficulty, or perhaps the incompatibility, incompetence, unwillingness, expense, uncertainty, inconvenience, or corruption of existing legal systems. By specifying a state-change system through a rich and unambiguous language, and furthermore architecting a system such that we can reasonably expect that an agreement will be thus enforced autonomously, we can provide a means to this end.

Dealings in this proposed system would have several attributes not often found in the real world. The incorruptibility of judgement, often difficult to find, comes naturally from a disinterested algorithmic interpreter. Transparency, or being able to see exactly how a state or judgement came about through the transaction log and rules or instructional codes, never happens perfectly in human-based systems since natural language is necessarily vague, information is often lacking, and plain old prejudices are difficult to shake.

Overall, we wish to provide a system such that users can be guaranteed that no matter with which other individuals, systems or organisations they interact, they can do so with absolute confidence in the possible outcomes and how those outcomes might come about.

\subsection{Previous Work} \label{ch:previous}

\cite{buterin2013ethereum} first proposed the kernel of this work in late November, 2013. Though now evolved in many ways, the key functionality of a block-chain with a Turing-complete language and an effectively unlimited inter-transaction storage capability remains unchanged.

\cite{dwork92pricingvia} provided the first work into the usage of a cryptographic proof of computational expenditure (``proof-of-work'') as a means of transmitting a value signal over the Internet. The value-signal was utilised here as a spam deterrence mechanism rather than any kind of currency, but critically demonstrated the potential for a basic data channel to carry a \textit{strong economic signal}, allowing a receiver to make a physical assertion without having to rely upon \textit{trust}. \cite{back2002hashcash} later produced a system in a similar vein.

The first example of utilising the proof-of-work as a strong economic signal to secure a currency was by \cite{vishnumurthy03karma:a}. In this instance, the token was used to keep peer-to-peer file transaction in check, providing ``consumers'' with the ability to make micro-payments to ``suppliers'' for their services. The security model afforded by the proof-of-work was augmented with digital signatures and a ledger in order to ensure that the historical record couldn't be corrupted and that malicious actors could not spoof payment or unjustly complain about service delivery. Five years later, \cite{nakamoto2008bitcoin} introduced another such proof-of-work-secured value token, somewhat wider in scope. The fruits of this project, Bitcoin, became the first widely adopted global decentralised transaction ledger.

Other projects built on Bitcoin's success; the alt-coins introduced numerous other currencies through alteration to the protocol. Some of the best known are Litecoin and Primecoin, discussed by \cite{sprankel2013technical}. Other projects sought to take the core value content mechanism of the protocol and repurpose it; \cite{aron2012bitcoin} discusses, for example, the Namecoin project which aims to provide a decentralised name-resolution system.

Other projects still aim to build upon the Bitcoin network itself, leveraging the large amount of value placed in the system and the vast amount of computation that goes into the consensus mechanism. The Mastercoin project, first proposed by \cite{mastercoin2013willett}, aims to build a richer protocol involving many additional high-level features on top of the Bitcoin protocol through utilisation of a number of auxiliary parts to the core protocol. The Coloured Coins project, proposed by \cite{colouredcoins2012rosenfeld}, takes a similar but more simplified strategy, embellishing the rules of a transaction in order to break the fungibility of Bitcoin's base currency and allow the creation and tracking of tokens through a special ``chroma-wallet''-protocol-aware piece of software.

Additional work has been done in the area with discarding the decentralisation foundation; Ripple, discussed by \cite{boutellier2014pirates}, has sought to create a ``federated'' system for currency exchange, effectively creating a new financial clearing system. It has demonstrated that high efficiency gains can be made if the decentralisation premise is discarded.

Early work on smart contracts has been done by \cite{szabo1997formalizing} and \cite{miller1997future}. Around the 1990s it became clear that algorithmic enforcement of agreements could become a significant force in human cooperation. Though no specific system was proposed to implement such a system, it was proposed that the future of law would be heavily affected by such systems. In this light, Ethereum may be seen as a general implementation of such a \textit{crypto-law} system.

%E language?
For a list of terms used in this paper, refer to Appendix \ref{ch:Terminology}.

\section{The Blockchain Paradigm} \label{ch:overview}

Ethereum, taken as a whole, can be viewed as a transaction-based state machine: we begin with a genesis state and incrementally execute transactions to morph it into some final state. It is this final state which we accept as the canonical ``version'' of the world of Ethereum. The state can include such information as account balances, reputations, trust arrangements, data pertaining to information of the physical world; in short, anything that can currently be represented by a computer is admissible. Transactions thus represent a valid arc between two states; the `valid' part is important---there exist far more invalid state changes than valid state changes. Invalid state changes might, \eg, be things such as reducing an account balance without an equal and opposite increase elsewhere. A valid state transition is one which comes about through a transaction. Formally:
\begin{equation}
\linkdest{Upsilon_state_transition}\linkdest{Upsilon}\boldsymbol{\sigma}_{t+1} \equiv \Upsilon(\boldsymbol{\sigma}_{t}, T)
\end{equation}

where $\Upsilon$ is the Ethereum state transition function. In Ethereum, $\Upsilon$, together with $\boldsymbol{\sigma}$ are considerably more powerful than any existing comparable system; $\Upsilon$ allows components to carry out arbitrary computation, while $\boldsymbol{\sigma}$ allows components to store arbitrary state between transactions.

Transactions are collated into blocks; blocks are chained together using a cryptographic hash as a means of reference. Blocks function as a journal, recording a series of transactions together with the previous block and an identifier for the final state (though do not store the final state itself---that would be far too big). They also punctuate the transaction series with incentives for nodes to \textit{mine}. This incentivisation takes place as a state-transition function, adding value to a nominated account.

Mining is the process of dedicating effort (working) to bolster one series of transactions (a block) over any other potential competitor block. It is achieved thanks to a cryptographically secure proof. This scheme is known as a proof-of-work and is discussed in detail in section \ref{ch:pow}.

Formally, we expand to:
\begin{eqnarray}
\boldsymbol{\sigma}_{t+1} & \equiv & \hyperlink{Pi}{\Pi}(\boldsymbol{\sigma}_{t}, B) \\
B & \equiv & (..., (T_0, T_1, ...), ...) \\
\Pi(\boldsymbol{\sigma}, B) & \equiv & \hyperlink{Omega}{\Omega}(B, \hyperlink{Upsilon}{\Upsilon}(\Upsilon(\boldsymbol{\sigma}, T_0), T_1) ...)
\end{eqnarray}

Where \hyperlink{Omega}{$\Omega$} is the block-finalisation state transition function (a function that rewards a nominated party); \hyperlink{block}{$B$} is this block, which includes a series of transactions amongst some other components; and $\hyperlink{Pi}{\Pi}$ is the block-level state-transition function.

This is the basis of the blockchain paradigm, a model that forms the backbone of not only Ethereum, but all decentralised consensus-based transaction systems to date.

\subsection{Value}

In order to incentivise computation within the network, there needs to be an agreed method for transmitting value. To address this issue, Ethereum has an intrinsic currency, Ether, known also as {\small ETH} and sometimes referred to by the Old English \DH{}. The smallest subdenomination of Ether, and thus the one in which all integer values of the currency are counted, is the Wei. One Ether is defined as being $10^{18}$ Wei. There exist other subdenominations of Ether:
\par
\begin{center}
\begin{tabular}{rl}
\toprule
Multiplier & Name \\
\midrule
$10^0$ & Wei \\
$10^{12}$ & Szabo \\
$10^{15}$ & Finney \\
$10^{18}$ & Ether \\
\bottomrule
\end{tabular}
\end{center}
\par

Throughout the present work, any reference to value, in the context of Ether, currency, a balance or a payment, should be assumed to be counted in Wei.

\subsection{Which History?}

Since the system is decentralised and all parties have an opportunity to create a new block on some older pre-existing block, the resultant structure is necessarily a tree of blocks. In order to form a consensus as to which path, from root (\hyperlink{Genesis_Block}{the genesis block}) to leaf (the block containing the most recent transactions) through this tree structure, known as the blockchain, there must be an agreed-upon scheme. If there is ever a disagreement between nodes as to which root-to-leaf path down the block tree is the `best' blockchain, then a \textit{fork} occurs.

This would mean that past a given point in time (block), multiple states of the system may coexist: some nodes believing one block to contain the canonical transactions, other nodes believing some other block to be canonical, potentially containing radically different or incompatible transactions. This is to be avoided at all costs as the uncertainty that would ensue would likely kill all confidence in the entire system.

The scheme we use in order to generate consensus is a simplified version of the GHOST protocol introduced by \cite{cryptoeprint:2013:881}. This process is described in detail in section \ref{ch:ghost}.

Sometimes, a path follows a new protocol from a particular height.  This document describes one version of the protocol.  In order to follow back the history of a path, one must reference multiple versions of this document.

\section{Conventions}\label{ch:conventions}

We use a number of typographical conventions for the formal notation, some of which are quite particular to the present work:

The two sets of highly structured, `top-level', state values, are denoted with bold lowercase Greek letters. They fall into those of world-state, which are denoted $\boldsymbol{\sigma}$ (or a variant thereupon) and those of machine-state, $\boldsymbol{\mu}$.

Functions operating on highly structured values are denoted with an upper-case Greek letter, \eg \hyperlink{Upsilon_state_transition}{$\Upsilon$}, the Ethereum state transition function.

For most functions, an uppercase letter is used, e.g. $C$, the general cost function. These may be subscripted to denote specialised variants, \eg \hyperlink{C__SSTORE}{$C_\text{SSTORE}$}, the cost function for the \hyperlink{SSTORE}{{\tiny SSTORE}} operation. For specialised and possibly externally defined functions, we may format as typewriter text, \eg the Keccak-256 hash function (as per the winning entry to the SHA-3 contest by \cite{Keccak}, rather than later releases), is denoted $\texttt{KEC}$ (and generally referred to as plain Keccak). Also $\texttt{KEC512}$ is referring to the Keccak 512 hash function.

Tuples are typically denoted with an upper-case letter, \eg $T$, is used to denote an Ethereum transaction. This symbol may, if accordingly defined, be subscripted to refer to an individual component, \eg \hyperlink{transaction_nonce}{$T_{\mathrm{n}}$}, denotes the nonce of said transaction. The form of the subscript is used to denote its type; \eg uppercase subscripts refer to tuples with subscriptable components.

Scalars and fixed-size byte sequences (or, synonymously, arrays) are denoted with a normal lower-case letter, \eg $n$ is used in the document to denote a \hyperlink{transaction_nonce}{transaction nonce}. Those with a particularly special meaning may be Greek, \eg $\delta$, the number of items required on the stack for a given operation.

Arbitrary-length sequences are typically denoted as a bold lower-case letter, \eg $\mathbf{o}$ is used to denote the byte sequence given as the output data of a message call. For particularly important values, a bold uppercase letter may be used.

Throughout, we assume scalars are non-negative integers and thus belong to the set $\mathbb{N}$. The set of all byte sequences is $\mathbb{B}$, formally defined in Appendix \ref{app:rlp}. If such a set of sequences is restricted to those of a particular length, it is denoted with a subscript, thus the set of all byte sequences of length $32$ is named $\mathbb{B}_{32}$ and the set of all non-negative integers smaller than $2^{256}$ is named $\mathbb{N}_{256}$. This is formally defined in section \hyperlink{block}{\ref{subsec:The_Block}}.

Square brackets are used to index into and reference individual components or subsequences of sequences, \eg $\boldsymbol{\mu}_{\mathbf{s}}[0]$ denotes the first item on the machine's stack. For subsequences, ellipses are used to specify the intended range, to include elements at both limits, \eg $\boldsymbol{\mu}_{\mathbf{m}}[0..31]$ denotes the first 32 items of the machine's memory.

In the case of the global state $\boldsymbol{\sigma}$, which is a sequence of accounts, themselves tuples, the square brackets are used to reference an individual account.

When considering variants of existing values, we follow the rule that within a given scope for definition, if we assume that the unmodified `input' value be denoted by the placeholder $\Box$ then the modified and utilisable value is denoted as $\Box'$, and intermediate values would be $\Box^*$,  $\Box^{**}$ \&c. On very particular occasions, in order to maximise readability and only if unambiguous in meaning, we may use alpha-numeric subscripts to denote intermediate values, especially those of particular note.

When considering the use of existing functions, given a function $f$, the function \hyperlink{general_element_wise_sequence_transformation_f_pow_asterisk}{$f^*$} denotes a similar, element-wise version of the function mapping instead between sequences. It is formally defined in section \hyperlink{block}{\ref{subsec:The_Block}}.

We define a number of useful functions throughout. \linkdest{ell}One of the more common is $\ell$, which evaluates to the last item in the given sequence:

\begin{equation}
\ell(\mathbf{x}) \equiv \mathbf{x}[\lVert \mathbf{x} \rVert - 1]
\end{equation}

\section{Blocks, State and Transactions} \label{ch:bst}

Having introduced the basic concepts behind Ethereum, we will discuss the meaning of a transaction, a block and the state in more detail.

\subsection{World State} \label{ch:state}

The world state (\textit{state}), is a mapping between addresses (160-bit identifiers) and account states (a data structure serialised as RLP, see Appendix \ref{app:rlp}). Though not stored on the blockchain, it is assumed that the implementation will maintain this mapping in a modified Merkle Patricia tree (\textit{trie}, see Appendix \ref{app:trie}). The trie requires a simple database backend that maintains a mapping of bytearrays to bytearrays; we name this underlying database the state database. This has a number of benefits; firstly the root node of this structure is cryptographically dependent on all internal data and as such its hash can be used as a secure identity for the entire system state. Secondly, being an immutable data structure, it allows any previous state (whose root hash is known) to be recalled by simply altering the root hash accordingly. Since we store all such root hashes in the blockchain, we are able to trivially revert to old states.

The account state, $\boldsymbol{\sigma}[a]$, comprises the following four fields:

\begin{description}
\item[nonce] \linkdest{account_nonce}A scalar value equal to the number of transactions sent from this address or, in the case of accounts with associated code, the number of contract-creations made by this account. For account of address $a$ in state $\boldsymbol{\sigma}$, this would be formally denoted $\boldsymbol{\sigma}[a]_{\mathrm{n}}$.
\item[balance] A scalar value equal to the number of Wei owned by this address. Formally denoted $\boldsymbol{\sigma}[a]_{\mathrm{b}}$.
\item[storageRoot] A 256-bit hash of the root node of a Merkle Patricia tree that encodes the storage contents of the account (a mapping between 256-bit integer values), encoded into the trie as a mapping from the Keccak 256-bit hash of the  256-bit integer keys to the RLP-encoded 256-bit integer values. The hash is formally denoted $\boldsymbol{\sigma}[a]_{\mathrm{s}}$.
\item[codeHash] The hash of the EVM code of this account---this is the code that gets executed should this address receive a message call; it is immutable and thus, unlike all other fields, cannot be changed after construction. All such code fragments are contained in the state database under their corresponding hashes for later retrieval. This hash is formally denoted $\boldsymbol{\sigma}[a]_{\mathrm{c}}$, and thus the code may be denoted as $\mathbf{b}$, given that $\texttt{KEC}(\mathbf{b}) = \boldsymbol{\sigma}[a]_{\mathrm{c}}$.
\end{description}

Since we typically wish to refer not to the trie's root hash but to the underlying set of key/value pairs stored within, we define a convenient equivalence:
\begin{equation}
\texttt{TRIE}\big(L_{I}^*(\boldsymbol{\sigma}[a]_{\mathbf{s}})\big) \equiv \boldsymbol{\sigma}[a]_{\mathrm{s}}
\end{equation}

The collapse function for the set of key/value pairs in the trie, $L_{I}^*$, is defined as the element-wise transformation of the base function $L_{I}$, given as:
\begin{equation}
L_{I}\big( (k, v) \big) \equiv \big(\texttt{KEC}(k), \texttt{RLP}(v)\big)
\end{equation}

where:
\begin{equation}
k \in \mathbb{B}_{32} \quad \wedge \quad v \in \mathbb{N}
\end{equation}

It shall be understood that $\boldsymbol{\sigma}[a]_{\mathbf{s}}$ is not a `physical' member of the account and does not contribute to its later serialisation.

If the \textbf{codeHash} field is the Keccak-256 hash of the empty string, i.e. $\boldsymbol{\sigma}[a]_{\mathrm{c}} = \texttt{KEC}\big(()\big)$, then the node represents a simple account, sometimes referred to as a ``non-contract'' account.

Thus we may define a world-state collapse function $L_{S}$:
\begin{equation}
L_{S}(\boldsymbol{\sigma}) \equiv \{ p(a): \boldsymbol{\sigma}[a] \neq \varnothing \}
\end{equation}
where
\begin{equation}
p(a) \equiv  \big(\texttt{KEC}(a), \texttt{RLP}\big( (\boldsymbol{\sigma}[a]_{\mathrm{n}}, \boldsymbol{\sigma}[a]_{\mathrm{b}}, \boldsymbol{\sigma}[a]_{\mathrm{s}}, \boldsymbol{\sigma}[a]_{\mathrm{c}}) \big) \big)
\end{equation}

This function, $L_{S}$, is used alongside the trie function to provide a short identity (hash) of the world state. We assume:
\begin{equation}
\forall a: \boldsymbol{\sigma}[a] = \varnothing \; \vee \; (a \in \mathbb{B}_{20} \; \wedge \; v(\boldsymbol{\sigma}[a]))
\end{equation}
\linkdest{account_validity_function_v__x}{}where $v$ is the account validity function:
\begin{equation}
\quad v(x) \equiv x_{\mathrm{n}} \in \mathbb{N}_{256} \wedge x_{\mathrm{b}} \in \mathbb{N}_{256} \wedge x_{\mathrm{s}} \in \mathbb{B}_{32} \wedge x_{\mathrm{c}} \in \mathbb{B}_{32}
\end{equation}

An account is \textit{empty} when it has no code, zero nonce and zero balance:
\begin{equation}
\mathtt{EMPTY}(\boldsymbol{\sigma}, a) \quad\equiv\quad \boldsymbol{\sigma}[a]_{\mathrm{c}} = \texttt{KEC}\big(()\big) \wedge \boldsymbol{\sigma}[a]_{\mathrm{n}} = 0 \wedge \boldsymbol{\sigma}[a]_{\mathrm{b}} = 0
\end{equation}
Even callable precompiled contracts can have an empty account state. This is because their account states do not usually contain the code describing its behavior.

An account is \textit{dead} when its account state is non-existent or empty:
\begin{equation}
\mathtt{DEAD}(\boldsymbol{\sigma}, a) \quad\equiv\quad \boldsymbol{\sigma}[a] = \varnothing \vee \mathtt{EMPTY}(\boldsymbol{\sigma}, a)
\end{equation}

\subsection{The Transaction} \label{subsec:transaction}

A transaction (formally, $T$) is a single cryptographically-signed instruction constructed by an actor externally to the scope of Ethereum. While it is assumed that the ultimate external actor will be human in nature, software tools will be used in its construction and dissemination\footnote{Notably, such `tools' could ultimately become so causally removed from their human-based initiation---or humans may become so causally-neutral---that there could be a point at which they rightly be considered autonomous agents. \eg contracts may offer bounties to humans for being sent transactions to initiate their execution.}. There are two types of transactions: those which result in message calls and those which result in the creation of new accounts with associated code (known informally as `contract creation'). Both types specify a number of common fields:

\begin{description}
\item[nonce]\linkdest{tx_nonce}{} A scalar value equal to the number of transactions sent by the sender; formally $T_{\mathrm{n}}$.
\item[gasPrice]\linkdest{tx_gas_price_T__p}{} A scalar value equal to the number of Wei to be paid per unit of \textit{gas} for all computation costs incurred as a result of the execution of this transaction; formally $T_{\mathrm{p}}$.
\item[gasLimit]\linkdest{tx_gas_limit_T__g}{} A scalar value equal to the maximum amount of gas that should be used in executing this transaction. This is paid up-front, before any computation is done and may not be increased later; formally $T_{\mathrm{g}}$.
\item[to]\linkdest{tx_to_address_T__t}{} The 160-bit address of the message call's recipient or, for a contract creation transaction, $\varnothing$, used here to denote the only member of $\mathbb{B}_0$ ; formally $T_{\mathrm{t}}$.
\item[value]\linkdest{tx_value_T__v}{} A scalar value equal to the number of Wei to be transferred to the message call's recipient or, in the case of contract creation, as an endowment to the newly created account; formally $T_{\mathrm{v}}$.
\item[v, r, s] Values corresponding to the signature of the transaction and used to determine the sender of the transaction; formally \linkdest{T__w_T__r_T__s}{$T_{\mathrm{w}}$, $T_{\mathrm{r}}$ and $T_{\mathrm{s}}$}. This is expanded in Appendix \ref{app:signing}.
\end{description}

Additionally, a contract creation transaction contains:

\begin{description}
\item[init] An unlimited size byte array specifying the EVM-code for the account initialisation procedure, formally $T_{\mathbf{i}}$.
\end{description}

\textbf{init} is an EVM-code fragment; it returns the \textbf{body}, a second fragment of code that executes each time the account receives a message call (either through a transaction or due to the internal execution of code). \textbf{init} is executed only once at account creation and gets discarded immediately thereafter.

In contrast, a message call transaction contains:

\begin{description}
\item[data] An unlimited size byte array specifying the input data of the message call, formally $T_{\mathbf{d}}$.
\end{description}

Appendix \ref{app:signing} specifies the function, $S$, which maps transactions to the sender, and happens through the ECDSA of the SECP-256k1 curve, using the hash of the transaction (excepting the latter three signature fields) as the datum to sign. For the present we simply assert that the sender of a given transaction $T$ can be represented with $S(T)$.

\begin{equation}
L_{T}(T) \equiv \begin{cases}
(T_{\mathrm{n}}, T_{\mathrm{p}}, T_{\mathrm{g}}, T_{\mathrm{t}}, T_{\mathrm{v}}, T_{\mathbf{i}}, T_{\mathrm{w}}, T_{\mathrm{r}}, T_{\mathrm{s}}) & \text{if} \; T_{\mathrm{t}} = \varnothing\\
(T_{\mathrm{n}}, T_{\mathrm{p}}, T_{\mathrm{g}}, T_{\mathrm{t}}, T_{\mathrm{v}}, T_{\mathbf{d}}, T_{\mathrm{w}}, T_{\mathrm{r}}, T_{\mathrm{s}}) & \text{otherwise}
\end{cases}
\end{equation}

Here, we assume all components are interpreted by the RLP as integer values, with the exception of the arbitrary length byte arrays $T_{\mathbf{i}}$ and $T_{\mathbf{d}}$.
\begin{equation}
\begin{array}[t]{lclclc}
T_{\mathrm{n}} \in \mathbb{N}_{256} & \wedge & T_{\mathrm{v}} \in \mathbb{N}_{256} & \wedge & T_{\mathrm{p}} \in \mathbb{N}_{256} & \wedge \\
T_{\mathrm{g}} \in \mathbb{N}_{256} & \wedge & T_{\mathrm{w}} \in \mathbb{N}_5 & \wedge & T_{\mathrm{r}} \in \mathbb{N}_{256} & \wedge \\
T_{\mathrm{s}} \in \mathbb{N}_{256} & \wedge & T_{\mathbf{d}} \in \mathbb{B} & \wedge & T_{\mathbf{i}} \in \mathbb{B}
\end{array}
\end{equation}
where
\begin{equation}
\mathbb{N}_{\mathrm{n}} = \{ P: P \in \mathbb{N} \wedge P < 2^n \}
\end{equation}

The address hash $T_{\mathbf{t}}$ is slightly different: it is either a 20-byte address hash or, in the case of being a contract-creation transaction (and thus formally equal to $\varnothing$), it is the RLP empty byte sequence and thus the member of $\mathbb{B}_0$:
\begin{equation}
T_{\mathbf{t}} \in \begin{cases} \mathbb{B}_{20} & \text{if} \quad T_{\mathrm{t}} \neq \varnothing \\
\mathbb{B}_{0} & \text{otherwise}\end{cases}
\end{equation}

\subsection{The Block}\linkdest{block}\label{subsec:The_Block}

The block in Ethereum is the collection of relevant pieces of information (known as the block \textit{header}), $H$, together with information corresponding to the comprised transactions, $\mathbf{T}$,\hypertarget{ommerheaders}{} and a set of other block headers $\mathbf{U}$ that are known to have a parent equal to the present block's parent's parent (such blocks are known as \textit{ommers}\footnote{\textit{ommer} is a gender-neutral term to mean ``sibling of parent''; see \url{https://nonbinary.miraheze.org/wiki/Gender_neutral_language\#Aunt.2FUncle}}). The block header contains several pieces of information:

%\textit{TODO: Introduce logs}

\begin{description}
\item[parentHash]\linkdest{parent_Hash_H__p_def_words}{} The Keccak 256-bit hash of the parent block's header, in its entirety; formally $H_{\mathrm{p}}$.
\item[ommersHash] The Keccak 256-bit hash of the ommers list portion of this block; formally $H_{\mathrm{o}}$.
\item[beneficiary]\linkdest{beneficiary_H__c}{}\linkdest{H__c} The 160-bit address to which all fees collected from the successful mining of this block be transferred; formally $H_{\mathrm{c}}$.
\item[stateRoot] The Keccak 256-bit hash of the root node of the state trie, after all transactions are executed and finalisations applied; formally $H_{\mathrm{r}}$.
\item[transactionsRoot] The Keccak 256-bit hash of the root node of the trie structure populated with each transaction in the transactions list portion of the block; formally $H_{\mathrm{t}}$.
\item[receiptsRoot]\linkdest{receipts_Root_def_words}{} The Keccak 256-bit hash of the root node of the trie structure populated with the receipts of each transaction in the transactions list portion of the block; formally $H_{\mathrm{e}}$.
\item[logsBloom]\linkdest{logs_Bloom_def_words}{} The Bloom filter composed from indexable information (logger address and log topics) contained in each log entry from the receipt of each transaction in the transactions list; formally $H_{\mathrm{b}}$.
\item[difficulty] A scalar value corresponding to the difficulty level of this block. This can be calculated from the previous block's difficulty level and the timestamp; formally $H_{\mathrm{d}}$.
\item[number]\linkdest{block_number_word_def_H_i}{} A scalar value equal to the number of ancestor blocks. The genesis block has a number of zero; formally \hyperlink{block_number_H__i}{$H_{\mathrm{i}}$}.
\item[gasLimit] A scalar value equal to the current limit of gas expenditure per block; formally $H_{\mathrm{l}}$.
\item[gasUsed]\linkdest{block_gas_used_H__g}{}\linkdest{H__g} A scalar value equal to the total gas used in transactions in this block; formally $H_{\mathrm{g}}$.
\item[timestamp]\linkdest{block_timestamp_word_def_H__s}{} A scalar value equal to the reasonable output of Unix's time() at this block's inception; formally \hyperlink{block_timestamp_H__s}{$H_{\mathrm{s}}$}.
\item[extraData]\linkdest{block_extraData_H__x}{} An arbitrary byte array containing data relevant to this block. This must be 32 bytes or fewer; formally $H_{\mathrm{x}}$.
\item[mixHash]\linkdest{mixHash_H__m}{}\linkdest{H__m} A 256-bit hash which, combined with the nonce, proves that a sufficient amount of computation has been carried out on this block; formally $H_{\mathrm{m}}$.
\item[nonce]\linkdest{block_nonce_H__n}{}\linkdest{block_nonce} A 64-bit value which, combined with the mix-hash, proves that a sufficient amount of computation has been carried out on this block; formally \hyperlink{H__n}{$H_{\mathrm{n}}$}.
\end{description}
\linkdest{ommer_block_headers_B__U}{}\linkdest{block_B}{}The other two components in the block are simply a list of ommer block headers (of the same format as above), $B_{\mathbf{U}}$ and a series of the transactions, $B_{\mathbf{T}}$. Formally, we can refer to a block $B$:
\begin{equation}
B \equiv (B_{H}, B_{\mathbf{T}}, B_{\mathbf{U}})
\end{equation}

\subsubsection{Transaction Receipt}\linkdest{Transaction_Receipt}{}

In order to encode information about a transaction concerning which it may be useful to form a zero-knowledge proof, or index and search, we encode a receipt of each transaction containing certain information from its execution.
Each receipt, denoted $B_{\mathbf{R}}[i]$ for the $i$th transaction, is placed in an index-keyed \hyperlink{trie}{trie} and the root recorded in the header as \hyperlink{Receipts_Root_H__e}{$H_{\mathrm{e}}$}.

\linkdest{transaction_receipt_R}{}\linkdest{tx_receipt_gas_used_R__u}{}\linkdest{R__u}The transaction receipt, $R$, is a tuple of four items comprising: the cumulative gas used in the block containing the transaction receipt as of immediately after the transaction has happened, $R_{\mathrm{u}}$, the set of logs created through execution of the transaction, \hyperlink{RLP_serialisation_of_a_sequence_of_other_items_R__l_math_def}{$R_\mathbf{l}$} and the Bloom filter composed from information in those logs, \hyperlink{RLP_serialisation_of_a_byte_array_R__b_math_def}{$R_{\mathrm{b}}$} and the status code of the transaction, $R_{\mathrm{z}}$:
\begin{equation}
R \equiv (R_{\mathrm{u}}, R_{\mathrm{b}}, R_{\mathbf{l}}, R_{\mathrm{z}})
\end{equation}

\hypertarget{transaction_receipt_preparation_function_for_RLP_serialisation}{}\linkdest{L__R}The function $L_{R}$ trivially prepares a transaction receipt for being transformed into an RLP-serialised byte array:
\begin{equation}
L_{R}(R) \equiv (0 \in \mathbb{B}_{256}, R_{\mathrm{u}}, R_{\mathrm{b}}, R_{\mathbf{l}})
\end{equation}
where $0 \in \mathbb{B}_{256}$ replaces the pre-transaction state root that existed in previous versions of the protocol.

\linkdest{R__z_assert}We assert that the status code $R_{\mathrm{z}}$ is a non-negative integer.
\begin{equation}
R_{\mathrm{z}} \in \mathbb{N}
\end{equation}

\linkdest{R__u_assert}We assert that $R_{\mathrm{u}}$, the cumulative gas used, is a non-negative integer and that the logs Bloom, $R_{\mathrm{b}}$, is a hash of size 2048 bits (256 bytes):
\begin{equation}
R_{\mathrm{u}} \in \mathbb{N} \quad \wedge \quad R_{\mathrm{b}} \in \mathbb{B}_{256}
\end{equation}

%Notably $B_{\mathbf{T}}$ does not get serialised into the block by the block preparation function $L_{B}$; it is merely a convenience equivalence.

The sequence $R_{\mathbf{l}}$ is a series of log entries, $(O_0, O_1, ...)$. A log entry, $O$, is a tuple of the logger's address, $O_a$, a possibly empty series of 32-byte log topics, $O_{\mathbf{t}}$ and some number of bytes of data, $O_{\mathbf{d}}$:
\begin{equation}
O \equiv (O_{\mathrm{a}}, ({O_{\mathbf{t}}}_0, {O_{\mathbf{t}}}_1, ...), O_{\mathbf{d}})
\end{equation}
\begin{equation}
O_{\mathrm{a}} \in \mathbb{B}_{20} \quad \wedge \quad \forall_{x \in O_{\mathbf{t}}}: x \in \mathbb{B}_{32} \quad \wedge \quad O_{\mathbf{d}} \in \mathbb{B}
\end{equation}

We define the Bloom filter function, $M$, to reduce a log entry into a single 256-byte hash:
\begin{equation}
M(O) \equiv \hyperlink{bigvee}{\bigvee}_{x \in \{O_{\mathrm{a}}\} \cup O_{\mathbf{t}}} \big( M_{3:2048}(x) \big)
\end{equation}

where $M_{3:2048}$ is a specialised Bloom filter that sets three bits out of 2048, given an arbitrary byte sequence. It does this through taking the low-order 11 bits of each of the first three pairs of bytes in a Keccak-256 hash of the byte sequence.\footnote{11 bits $= 2^{2048}$, and the low-order 11 bits is the modulo 2048 of the operand, which is in this case is "each of the first three pairs of bytes in a Keccak-256 hash of the byte sequence."} Formally:
\begin{eqnarray}
M_{3:2048}(\mathbf{x}: \mathbf{x} \in \mathbb{B}) & \equiv & \mathbf{y}: \mathbf{y} \in \mathbb{B}_{256} \quad \text{where:}\\
\mathbf{y} & = & (0, 0, ..., 0) \quad \text{except:}\\
\forall_{i \in \{0, 2, 4\}}&:& \mathcal{B}_{m(\mathbf{x}, i)}(\mathbf{y}) = 1\\
m(\mathbf{x}, i) &\equiv& \mathtt{KEC}(\mathbf{x})[i, i + 1] \bmod 2048
\end{eqnarray}

where $\mathcal{B}$ is the bit reference function such that $\mathcal{B}_{\mathrm{j}}(\mathbf{x})$ equals the bit of index $j$ (indexed from 0) in the byte array $\mathbf{x}$.

\subsubsection{Holistic Validity}

\linkdest{block_validity}{}We can assert a block's validity if and only if it satisfies several conditions: it must be internally consistent with the ommer and transaction block hashes and the given transactions $B_{\mathbf{T}}$ (as specified in sec \ref{ch:finalisation}), when executed in order on the base state $\boldsymbol{\sigma}$ (derived from the final state of the parent block), result in a new state of the identity $H_{\mathrm{r}}$:
\begin{equation}
\begin{array}[t]{lclc}
\linkdest{new_state_H__r}{}H_{\mathrm{r}} &\equiv& \mathtt{TRIE}(L_S(\Pi(\boldsymbol{\sigma}, B))) & \wedge \\
\linkdest{Ommer_block_hash_H__o}{}H_{\mathrm{o}} &\equiv& \mathtt{KEC}(\mathtt{RLP}(L_H^*(B_{\mathbf{U}}))) & \wedge \\
\linkdest{tx_block_hash_H__t}{}H_{\mathrm{t}} &\equiv& \mathtt{TRIE}(\{\forall i < \lVert B_{\mathbf{T}} \rVert, i \in \mathbb{N}: &\\&& \quad\quad p (i, L_{T}(B_{\mathbf{T}}[i]))\}) & \wedge \\
\linkdest{Receipts_Root_H__e}{}H_{\mathrm{e}} &\equiv& \mathtt{TRIE}(\{\forall i < \lVert B_{\mathbf{R}} \rVert, i \in \mathbb{N}: &\\&& \quad\quad p(i, \hyperlink{transaction_receipt_preparation_function_for_RLP_serialisation}{L_{R}}(B_{\mathbf{R}}[i]))\}) & \wedge \\
\linkdest{logs_Bloom_filter_H__b}{}H_{\mathrm{b}} &\equiv& \bigvee_{\mathbf{r} \in B_{\mathbf{R}}} \big( \mathbf{r}_{\mathrm{b}} \big)
\end{array}
\end{equation}
where $p(k, v)$ is simply the pairwise RLP transformation, in this case, the first being the index of the transaction in the block and the second being the transaction receipt:
\begin{equation}
p(k, v) \equiv \big( \mathtt{RLP}(k), \mathtt{RLP}(v) \big)
\end{equation}

Furthermore:
\begin{equation}
\mathtt{TRIE}(L_{S}(\boldsymbol{\sigma})) = {P(B_H)_H}_{\mathrm{r}}
\end{equation}

Thus $\texttt{TRIE}(L_{S}(\boldsymbol{\sigma}))$ is the root node hash of the Merkle Patricia tree structure containing the key-value pairs of the state $\boldsymbol{\sigma}$ with values encoded using RLP, and $P(B_{H})$ is the parent block of $B$, defined directly.

The values stemming from the computation of transactions, specifically the \hyperlink{Transaction_Receipt}{transaction receipts}, $B_{\mathbf{R}}$, and that defined through the transaction's \hyperlink{Pi}{state-accumulation function, $\Pi$}, are formalised later in section \ref{sec:statenoncevalidation}.

\subsubsection{Serialisation}

\hypertarget{block_preparation_function_for_RLP_serialization_L__B}{}\linkdest{L__B}\hypertarget{block_preparation_function_for_RLP_serialization_L__H}{}\linkdest{L__B}The function $L_{B}$ and $L_{H}$ are the preparation functions for a block and block header respectively. Much like the \hyperlink{transaction_receipt_preparation_function_for_RLP_serialisation}{transaction receipt preparation function $L_{R}$}, we assert the types and order of the structure for when the RLP transformation is required:
\begin{eqnarray}
\quad L_{H}(H) & \equiv & (\begin{array}[t]{l}H_{\mathrm{p}}, H_{\mathrm{o}}, H_{\mathrm{c}}, H_{\mathrm{r}}, H_{\mathrm{t}}, H_{\mathrm{e}}, H_{\mathrm{b}}, H_{\mathrm{d}},\\ H_{\mathrm{i}}, H_{\mathrm{l}}, H_{\mathrm{g}}, H_{\mathrm{s}}, H_{\mathrm{x}}, H_{\mathrm{m}}, H_{\mathrm{n}} \; )\end{array} \\
\quad L_{B}(B) & \equiv & \big( L_{H}(B_{H}), L_{T}^*(B_{\mathbf{T}}), L_{H}^*(\hyperlink{ommer_block_headers_B__U}{B_{\mathbf{U}}}) \big)
\end{eqnarray}

\hypertarget{general_element_wise_sequence_transformation_f_pow_asterisk}{}With $L_T^*$ and $L_H^*$ being element-wise sequence transformations, thus:
\begin{equation}
\hyperlink{general_element_wise_sequence_transformation_f_pow_asterisk}{f^*}\big( (x_0, x_1, ...) \big) \equiv \big( f(x_0), f(x_1), ... \big) \quad \text{for any function} \; f
\end{equation}

The component types are defined thus:
\begin{equation}
\begin{array}[t]{lclclcl}
\hyperlink{parent_Hash_H__p_def_words}{H_{\mathrm{p}}} \in \mathbb{B}_{32} & \wedge & H_{\mathrm{o}} \in \mathbb{B}_{32} & \wedge & H_{\mathrm{c}} \in \mathbb{B}_{20} & \wedge \\
\hyperlink{new_state_H__r}{H_{\mathrm{r}}} \in \mathbb{B}_{32} & \wedge & H_{\mathrm{t}} \in \mathbb{B}_{32} & \wedge & \hyperlink{Receipts_Root_H__e}{H_{\mathrm{e}}} \in \mathbb{B}_{32} & \wedge \\
\hyperlink{logs_Bloom_filter_H__b}{H_{\mathrm{b}}} \in \mathbb{B}_{256} & \wedge & H_{\mathrm{d}} \in \mathbb{N} & \wedge & \hyperlink{block_number_H__i}{H_{\mathrm{i}}} \in \mathbb{N} & \wedge \\
\hyperlink{block_gas_limit_H__l}{H_{\mathrm{l}}} \in \mathbb{N} & \wedge & H_{\mathrm{g}} \in \mathbb{N} & \wedge & \hyperlink{block_timestamp_H__s}{H_{\mathrm{s}}} \in \mathbb{N}_{256} & \wedge \\
\hyperlink{block_extraData_H__x}{H_{\mathrm{x}}} \in \mathbb{B} & \wedge & H_{\mathrm{m}} \in \mathbb{B}_{32} & \wedge & \hyperlink{block_nonce_H__n}{H_{\mathrm{n}}} \in \mathbb{B}_{8}
\end{array}
\end{equation}

where
\begin{equation}
\mathbb{B}_{\mathrm{n}} = \{ B: B \in \mathbb{B} \wedge \lVert B \rVert = n \}
\end{equation}

We now have a rigorous specification for the construction of a formal block structure. The RLP function $\texttt{RLP}$ (see Appendix \ref{app:rlp}) provides the canonical method for transforming this structure into a sequence of bytes ready for transmission over the wire or storage locally.

\subsubsection{Block Header Validity}

We define $P(B_{H})$ to be the parent block of $B$, formally:
\begin{equation}
P(H) \equiv B': \mathtt{KEC}(\mathtt{RLP}(B'_{H})) = \hyperlink{parent_Hash_H__p_def_words}{H_{\mathrm{p}}}
\end{equation}

\hypertarget{block_number_H__i}{}The block number is the parent's block number incremented by one:
\begin{equation}
H_{\mathrm{i}} \equiv {{P(H)_{H}}_{\mathrm{i}}} + 1
\end{equation}

\newcommand{\mindifficulty}{D_0}
\newcommand{\homesteadmod}{\ensuremath{\varsigma_2}}
\newcommand{\expdiffsymb}{\ensuremath{\epsilon}}
\newcommand{\diffadjustment}{x}

\hypertarget{block_difficulty_H__d}{}\linkdest{H__d}The canonical difficulty of a block of header $H$ is defined as $D(H)$:
\begin{equation}
D(H) \equiv \begin{dcases}
\mindifficulty & \text{if} \quad H_{\mathrm{i}} = 0\\
\text{max}\!\left(\mindifficulty, {P(H)_{H}}_{\mathrm{d}} + \diffadjustment\times\homesteadmod + \expdiffsymb \right) & \text{otherwise}\\
\end{dcases}
\end{equation}
where:
\begin{equation}
\mindifficulty \equiv 131072
\end{equation}
\begin{equation}
\diffadjustment \equiv \left\lfloor\frac{{P(H)_{H}}_{\mathrm{d}}}{2048}\right\rfloor
\end{equation}
\begin{equation}
\homesteadmod \equiv \text{max}\left( y - \left\lfloor\frac{H_{\mathrm{s}} - {P(H)_{H}}_{\mathrm{s}}}{9}\right\rfloor, -99 \right)
\end{equation}
\begin{equation*}
y \equiv \begin{cases}
1 & \text{if} \, \lVert P(H)_{\mathbf{U}}\rVert = 0 \\
2 & \text{otherwise}
\end{cases}
\end{equation*}
\begin{align}
\expdiffsymb &\equiv \left\lfloor 2^{ \left\lfloor H'_{\mathrm