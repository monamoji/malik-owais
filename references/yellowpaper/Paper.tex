\documentclass[9pt,oneside]{amsart}
%\usepackage{tweaklist}
\usepackage{cancel}
\usepackage{xspace}
\usepackage{graphicx}
\usepackage{multicol}
\usepackage{subfig}
\usepackage{amsmath}
\usepackage{amssymb}
\usepackage[a4paper,width=170mm,top=18mm,bottom=22mm,includeheadfoot]{geometry}
\usepackage{booktabs}
\usepackage{array}
\usepackage{verbatim}
\usepackage{caption}
\usepackage{natbib}
\usepackage{float}
\usepackage{pdflscape}
\usepackage{mathtools}
\usepackage[usenames,dvipsnames]{xcolor}
\usepackage{afterpage}
\usepackage{tikz}
\usepackage[bookmarks=true, unicode=true, pdftitle={Ethereum Yellow Paper: a formal specification of Ethereum, a programmable blockchain}, pdfauthor={Dr. Gavin Wood},pdfkeywords={Ethereum, Yellow Paper, blockchain, virtual machine, cryptography, decentralised, singleton, transaction, generalised},pdfborder={0 0 0.5 [1 3]}]{hyperref}
%,pagebackref=true

\usepackage{tabu} %requires array.

%This should be the last package before \input{Version.tex}
\PassOptionsToPackage{hyphens}{url}\usepackage{hyperref}
% "hyperref loads the url package internally. Use \PassOptionsToPackage{hyphens}{url}\usepackage{hyperref} to pass the option to the url package when it is loaded by hyperref. This avoids any package option clashes." Source: <https://tex.stackexchange.com/questions/3033/forcing-linebreaks-in-url/3034#comment44478_3034>.
% Note also this: "If the \PassOptionsToPackage{hyphens}{url} approach does not work, maybe it's "because you're trying to load the url package with a specific option, but it's being loaded by one of your packages before that with a different set of options. Try loading the url package earlier than the package that requires it. If it's loaded by the document class, try using \RequirePackage[hyphens]{url} before the document class." Source: <https://tex.stackexchange.com/questions/3033/forcing-linebreaks-in-url/3034#comment555944_3034>.
% For more information on using the hyperref package, refer to e.g. https://en.wikibooks.org/w/index.php?title=LaTeX/Hyperlinks&stable=0#Hyperlink_and_Hypertarget.

\makeatletter
 \newcommand{\linkdest}[1]{\Hy@raisedlink{\hypertarget{#1}{}}}
\makeatother
\usepackage{seqsplit}

% For formatting
%\usepackage{underscore}
%\usepackage{lipsum} % to generate filler text for testing of document rendering
\usepackage[english]{babel}
\usepackage[autostyle]{csquotes}
\MakeOuterQuote{"}

\usepackage[final]{microtype} % https://tex.stackexchange.com/questions/75140/is-it-possible-to-make-latex-mark-overfull-boxes-in-the-output#comment382776_75142

\input{Version.tex}
% Default rendering options
\definecolor{pagecolor}{rgb}{1,0.98,0.9}
\def\YellowPaperVersionNumber{unknown revision}
\IfFileExists{Options.tex}{\input{Options.tex}}

\newcommand{\hcancel}[1]{%
    \tikz[baseline=(tocancel.base)]{
        \node[inner sep=0pt,outer sep=0pt] (tocancel) {#1};
        \draw[black] (tocancel.south west) -- (tocancel.north east);
    }%
}%


\DeclarePairedDelimiter{\ceil}{\lceil}{\rceil}
\newcommand*\eg{e.g.\@\xspace}
\newcommand*\Eg{e.g.\@\xspace}
\newcommand*\ie{i.e.\@\xspace}
%\renewcommand{\itemhook}{\setlength{\topsep}{0pt}  \setlength{\itemsep}{0pt}\setlength{\leftmargin}{15pt}}

\title{Ethereum: A Secure Decentralised Generalised Transaction Ledger \\ {\smaller \textbf{Byzantium version \YellowPaperVersionNumber}}}
\author{
    Dr. Gavin Wood\\
    Founder, Ethereum \& Parity\\
    gavin@parity.io
}
\begin{document}

\pagecolor{pagecolor}

\begin{abstract}
The blockchain paradigm when coupled with cryptographically-secured transactions has demonstrated its utility through a number of projects, with Bitcoin being one of the most notable ones. Each such project can be seen as a simple application on a decentralised, but singleton, compute resource. We can call this paradigm a transactional singleton machine with shared-state.

Ethereum implements this paradigm in a generalised manner. Furthermore it provides a plurality of such resources, each with a distinct state and operating code but able to interact through a message-passing framework with others. We discuss its design, implementation issues, the opportunities it provides and the future hurdles we envisage.
\end{abstract}

\maketitle

\setlength{\columnsep}{20pt}
\begin{multicols}{2}

\section{Introduction}\label{sec:introduction}

With ubiquitous internet connections in most places of the world, global information transmission has become incredibly cheap. Technology-rooted movements like Bitcoin have demonstrated through the power of the default, consensus mechanisms, and voluntary respect of the social contract, that it is possible to use the internet to make a decentralised value-transfer system that can be shared across the world and virtually free to use. This system can be said to be a very specialised version of a cryptographically secure, transaction-based state machine. Follow-up systems such as Namecoin adapted this original ``currency application'' of the technology into other applications albeit rather simplistic ones.

Ethereum is a project which attempts to build the generalised technology; technology on which all transaction-based state machine concepts may be built. Moreover it aims to provide to the end-developer a tightly integrated end-to-end system for building software on a hitherto unexplored compute paradigm in the mainstream: a trustful object messaging compute framework.

\subsection{Driving Factors} \label{ch:driving}

There are many goals of this project; one key goal is to facilitate transactions between consenting individuals who would otherwise have no means to trust one another. This may be due to geographical separation, interfacing difficulty, or perhaps the incompatibility, incompetence, unwillingness, expense, uncertainty, inconvenience, or corruption of existing legal systems. By specifying a state-change system through a rich and unambiguous language, and furthermore architecting a system such that we can reasonably expect that an agreement will be thus enforced autonomously, we can provide a means to this end.

Dealings in this proposed system would have several attributes not often found in the real world. The incorruptibility of judgement, often difficult to find, comes naturally from a disinterested algorithmic interpreter. Transparency, or being able to see exactly how a state or judgement came about through the transaction log and rules or instructional codes, never happens perfectly in human-based systems since natural language is necessarily vague, information is often lacking, and plain old prejudices are difficult to shake.

Overall, we wish to provide a system such that users can be guaranteed that no matter with which other individuals, systems or organisations they interact, they can do so with absolute confidence in the possible outcomes and how those outcomes might come about.

\subsection{Previous Work} \label{ch:previous}

\cite{buterin2013ethereum} first proposed the kernel of this work in late November, 2013. Though now evolved in many ways, the key functionality of a block-chain with a Turing-complete language and an effectively unlimited inter-transaction storage capability remains unchanged.

\cite{dwork92pricingvia} provided the first work into the usage of a cryptographic proof of computational expenditure (``proof-of-work'') as a means of transmitting a value signal over the Internet. The value-signal was utilised here as a spam deterrence mechanism rather than any kind of currency, but critically demonstrated the potential for a basic data channel to carry a \textit{strong economic signal}, allowing a receiver to make a physical assertion without having to rely upon \textit{trust}. \cite{back2002hashcash} later produced a system in a similar vein.

The first example of utilising the proof-of-work as a strong economic signal to secure a currency was by \cite{vishnumurthy03karma:a}. In this instance, the token was used to keep peer-to-peer file transaction in check, providing ``consumers'' with the ability to make micro-payments to ``suppliers'' for their services. The security model afforded by the proof-of-work was augmented with digital signatures and a ledger in order to ensure that the historical record couldn't be corrupted and that malicious actors could not spoof payment or unjustly complain about service delivery. Five years later, \cite{nakamoto2008bitcoin} introduced another such proof-of-work-secured value token, somewhat wider in scope. The fruits of this project, Bitcoin, became the first widely adopted global decentralised transaction ledger.

Other projects built on Bitcoin's success; the alt-coins introduced numerous other currencies through alteration to the protocol. Some of the best known are Litecoin and Primecoin, discussed by \cite{sprankel2013technical}. Other projects sought to take the core value content mechanism of the protocol and repurpose it; \cite{aron2012bitcoin} discusses, for example, the Namecoin project which aims to provide a decentralised name-resolution system.

Other projects still aim to build upon the Bitcoin network itself, leveraging the large amount of value placed in the system and the vast amount of computation that goes into the consensus mechanism. The Mastercoin project, first proposed by \cite{mastercoin2013willett}, aims to build a richer protocol involving many additional high-level features on top of the Bitcoin protocol through utilisation of a number of auxiliary parts to the core protocol. The Coloured Coins project, proposed by \cite{colouredcoins2012rosenfeld}, takes a similar but more simplified strategy, embellishing the rules of a transaction in order to break the fungibility of Bitcoin's base currency and allow the creation and tracking of tokens through a special ``chroma-wallet''-protocol-aware piece of software.

Additional work has been done in the area with discarding the decentralisation foundation; Ripple, discussed by \cite{boutellier2014pirates}, has sought to create a ``federated'' system for currency exchange, effectively creating a new financial clearing system. It has demonstrated that high efficiency gains can be made if the decentralisation premise is discarded.

Early work on smart contracts has been done by \cite{szabo1997formalizing} and \cite{miller1997future}. Around the 1990s it became clear that algorithmic enforcement of agreements could become a significant force in human cooperation. Though no specific system was proposed to implement such a system, it was proposed that the future of law would be heavily affected by such systems. In this light, Ethereum may be seen as a general implementation of such a \textit{crypto-law} system.

%E language?
For a list of terms used in this paper, refer to Appendix \ref{ch:Terminology}.

\section{The Blockchain Paradigm} \label{ch:overview}

Ethereum, taken as a whole, can be viewed as a transaction-based state machine: we begin with a genesis state and incrementally execute transactions to morph it into some final state. It is this final state which we accept as the canonical ``version'' of the world of Ethereum. The state can include such information as account balances, reputations, trust arrangements, data pertaining to information of the physical world; in short, anything that can currently be represented by a computer is admissible. Transactions thus represent a valid arc between two states; the `valid' part is important---there exist far more invalid state changes than valid state changes. Invalid state changes might, \eg, be things such as reducing an account balance without an equal and opposite increase elsewhere. A valid state transition is one which comes about through a transaction. Formally:
\begin{equation}
\linkdest{Upsilon_state_transition}\linkdest{Upsilon}\boldsymbol{\sigma}_{t+1} \equiv \Upsilon(\boldsymbol{\sigma}_{t}, T)
\end{equation}

where $\Upsilon$ is the Ethereum state transition function. In Ethereum, $\Upsilon$, together with $\boldsymbol{\sigma}$ are considerably more powerful than any existing comparable system; $\Upsilon$ allows components to carry out arbitrary computation, while $\boldsymbol{\sigma}$ allows components to store arbitrary state between transactions.

Transactions are collated into blocks; blocks are chained together using a cryptographic hash as a means of reference. Blocks function as a journal, recording a series of transactions together with the previous block and an identifier for the final state (though do not store the final state itself---that would be far too big). They also punctuate the transaction series with incentives for nodes to \textit{mine}. This incentivisation takes place as a state-transition function, adding value to a nominated account.

Mining is the process of dedicating effort (working) to bolster one series of transactions (a block) over any other potential competitor block. It is achieved thanks to a cryptographically secure proof. This scheme is known as a proof-of-work and is discussed in detail in section \ref{ch:pow}.

Formally, we expand to:
\begin{eqnarray}
\boldsymbol{\sigma}_{t+1} & \equiv & \hyperlink{Pi}{\Pi}(\boldsymbol{\sigma}_{t}, B) \\
B & \equiv & (..., (T_0, T_1, ...), ...) \\
\Pi(\boldsymbol{\sigma}, B) & \equiv & \hyperlink{Omega}{\Omega}(B, \hyperlink{Upsilon}{\Upsilon}(\Upsilon(\boldsymbol{\sigma}, T_0), T_1) ...)
\end{eqnarray}

Where \hyperlink{Omega}{$\Omega$} is the block-finalisation state transition function (a function that rewards a nominated party); \hyperlink{block}{$B$} is this block, which includes a series of transactions amongst some other components; and $\hyperlink{Pi}{\Pi}$ is the block-level state-transition function.

This is the basis of the blockchain paradigm, a model that forms the backbone of not only Ethereum, but all decentralised consensus-based transaction systems to date.

\subsection{Value}

In order to incentivise computation within the network, there needs to be an agreed method for transmitting value. To address this issue, Ethereum has an intrinsic currency, Ether, known also as {\small ETH} and sometimes referred to by the Old English \DH{}. The smallest subdenomination of Ether, and thus the one in which all integer values of the currency are counted, is the Wei. One Ether is defined as being $10^{18}$ Wei. There exist other subdenominations of Ether:
\par
\begin{center}
\begin{tabular}{rl}
\toprule
Multiplier & Name \\
\midrule
$10^0$ & Wei \\
$10^{12}$ & Szabo \\
$10^{15}$ & Finney \\
$10^{18}$ & Ether \\
\bottomrule
\end{tabular}
\end{center}
\par

Throughout the present work, any reference to value, in the context of Ether, currency, a balance or a payment, should be assumed to be counted in Wei.

\subsection{Which History?}

Since the system is decentralised and all parties have an opportunity to create a new block on some older pre-existing block, the resultant structure is necessarily a tree of blocks. In order to form a consensus as to which path, from root (\hyperlink{Genesis_Block}{the genesis block}) to leaf (the block containing the most recent transactions) through this tree structure, known as the blockchain, there must be an agreed-upon scheme. If there is ever a disagreement between nodes as to which root-to-leaf path down the block tree is the `best' blockchain, then a \textit{fork} occurs.

This would mean that past a given point in time (block), multiple states of the system may coexist: some nodes believing one block to contain the canonical transactions, other nodes believing some other block to be canonical, potentially containing radically different or incompatible transactions. This is to be avoided at all costs as the uncertainty that would ensue would likely kill all confidence in the entire system.

The scheme we use in order to generate consensus is a simplified version of the GHOST protocol introduced by \cite{cryptoeprint:2013:881}. This process is described in detail in section \ref{ch:ghost}.

Sometimes, a path follows a new protocol from a particular height.  This document describes one version of the protocol.  In order to follow back the history of a path, one must reference multiple versions of this document.

\section{Conventions}\label{ch:conventions}

We use a number of typographical conventions for the formal notation, some of which are quite particular to the present work:

The two sets of highly structured, `top-level', state values, are denoted with bold lowercase Greek letters. They fall into those of world-state, which are denoted $\boldsymbol{\sigma}$ (or a variant thereupon) and those of machine-state, $\boldsymbol{\mu}$.

Functions operating on highly structured values are denoted with an upper-case Greek letter, \eg \hyperlink{Upsilon_state_transition}{$\Upsilon$}, the Ethereum state transition function.

For most functions, an uppercase letter is used, e.g. $C$, the general cost function. These may be subscripted to denote specialised variants, \eg \hyperlink{C__SSTORE}{$C_\text{SSTORE}$}, the cost function for the \hyperlink{SSTORE}{{\tiny SSTORE}} operation. For specialised and possibly externally defined functions, we may format as typewriter text, \eg the Keccak-256 hash function (as per the winning entry to the SHA-3 contest by \cite{Keccak}, rather than later releases), is denoted $\texttt{KEC}$ (and generally referred to as plain Keccak). Also $\texttt{KEC512}$ is referring to the Keccak 512 hash function.

Tuples are typically denoted with an upper-case letter, \eg $T$, is used to denote an Ethereum transaction. This symbol may, if accordingly defined, be subscripted to refer to an individual component, \eg \hyperlink{transaction_nonce}{$T_{\mathrm{n}}$}, denotes the nonce of said transaction. The form of the subscript is used to denote its type; \eg uppercase subscripts refer to tuples with subscriptable components.

Scalars and fixed-size byte sequences (or, synonymously, arrays) are denoted with a normal lower-case letter, \eg $n$ is used in the document to denote a \hyperlink{transaction_nonce}{transaction nonce}. Those with a particularly special meaning may be Greek, \eg $\delta$, the number of items required on the stack for a given operation.

Arbitrary-length sequences are typically denoted as a bold lower-case letter, \eg $\mathbf{o}$ is used to denote the byte sequence given as the output data of a message call. For particularly important values, a bold uppercase letter may be used.

Throughout, we assume scalars are non-negative integers and thus belong to the set $\mathbb{N}$. The set of all byte sequences is $\mathbb{B}$, formally defined in Appendix \ref{app:rlp}. If such a set of sequences is restricted to those of a particular length, it is denoted with a subscript, thus the set of all byte sequences of length $32$ is named $\mathbb{B}_{32}$ and the set of all non-negative integers smaller than $2^{256}$ is named $\mathbb{N}_{256}$. This is formally defined in section \hyperlink{block}{\ref{subsec:The_Block}}.

Square brackets are used to index into and reference individual components or subsequences of sequences, \eg $\boldsymbol{\mu}_{\mathbf{s}}[0]$ denotes the first item on the machine's stack. For subsequences, ellipses are used to specify the intended range, to include elements at both limits, \eg $\boldsymbol{\mu}_{\mathbf{m}}[0..31]$ denotes the first 32 items of the machine's memory.

In the case of the global state $\boldsymbol{\sigma}$, which is a sequence of accounts, themselves tuples, the square brackets are used to reference an individual account.

When considering variants of existing values, we follow the rule that within a given scope for definition, if we assume that the unmodified `input' value be denoted by the placeholder $\Box$ then the modified and utilisable value is denoted as $\Box'$, and in